% !TEX encoding = UTF-8 Unicode
% !TEX root = ../m2report.tex

The \InstGen framework by Ganzinger and Korovin is a powerfull instrument for instantiation-based
first order theorem proving
that relies on the efficiency of satisfiability checkers.
This framework has been extended to \InstGenEq with the unit superposition calculus for equational reasoning.
Sticksel's implementation of this extension has clearly demonstrated its practical power,
but has also highlighted that the calculus is a heavy and crucial addition that does not benefit
from the sophistication of \SAT solving. 
%This added a heavy and crucial part that does not use sat solving.
Maximal completion by Klein and Hirokawa is a simple and smart approach for completion of equational systems
that exploits the impressive performance of maximal satisfiability solvers.
This report looks into some possibilities and obstacles to use maximal completion 
for equational reasoning together with the \InstGen framework
in the context of an instantiation-based first order theorem prover with equality.

%This is a programmers naive approach to explore the applicability of maximal completion in instantiation-based first order theorem proving with equality.
